\section{远程分支创建与提交全流程指南}
\textbf{一句话总结:}  
通过本地分支关联远程仓库实现分支创建与提交,核心流程为:本地创建分支 → 推送至远程 → 开发提交 → 定期同步。

\subsection{完整操作流程}
\begin{enumerate}[leftmargin=*, nosep]
    \item \textbf{创建本地分支} \\
    {\color{red}\texttt{\$ git checkout -b feature/login}}\quad \textcolor{gray}{// 创建并切换到新分支}
    
    \item \textbf{关联远程仓库} \\
   {\color{red} \texttt{\$ git push -u origin feature/login} }\\
    \textcolor{gray}{→ 推送分支到远程并建立追踪关系(首次推送)}
    
    \item \textbf{开发并提交代码} \\
    \texttt{\$ touch login.js} \\
    {\color{red} \texttt{\$ git add login.js} }\\
    {\color{red} \texttt{\$ git commit -m "添加登录页面"} }\\
    {\color{red} \texttt{\$ git push} \quad \textcolor{gray}{// 后续提交只需git push}}
    
    \item \textbf{同步远程更新} \\
    \texttt{\$ git pull origin feature/login} \quad \textcolor{gray}{// 获取他人提交}
\end{enumerate}

\subsection{关键步骤详解}
\subsubsection{1. 创建本地分支}
\begin{itemize}[leftmargin=*, nosep]
    \item \textbf{基于当前分支创建}:\\
    \texttt{git branch <新分支名>} + \texttt{git checkout <新分支名>}
    
    \item \textbf{基于特定提交创建}:\\
    \texttt{git checkout -b fix/issue-23 8a3f5e9}
    
    \item \textbf{克隆远程分支}:\\
    \texttt{git checkout -b dev origin/dev}
\end{itemize}

\subsubsection{2. 推送至远程仓库}
\begin{center}
\begin{tabular}{@{}ll@{}}
    \toprule
    \textbf{命令} & \textbf{作用} \\
    \midrule
    \texttt{git push -u origin <分支名>} & 首次推送并建立追踪 \\
    \texttt{git push origin <分支名>} & 后续推送更新 \\
    \texttt{git push --set-upstream origin <分支名>} & 为现有分支设置上游 \\
    \texttt{git push -f} & 强制覆盖(慎用) \\
    \bottomrule
\end{tabular}
\end{center}

\subsubsection{3. 验证远程分支}
\begin{itemize}[leftmargin=*, nosep]
    \item \textbf{查看远程分支}:\\
    \texttt{git branch -r} \quad \textcolor{gray}{// 查看远程分支列表}
    
    \item \textbf{检查关联状态}:\\
    \texttt{git branch -vv} \\
    \textcolor{gray}{→ 显示 \texttt{feature/login [origin/feature/login]}}
\end{itemize}

\subsection{分支状态管理}
\begin{center}
\begin{tabular}{@{}llp{8cm}@{}}
    \toprule
    \textbf{状态} & \textbf{检测命令} & \textbf{处理方案} \\
    \midrule
    本地领先 & \texttt{git status} 提示"ahead" & 立即推送 \\
    远程领先 & \texttt{git status} 提示"behind" & 先执行 \texttt{git pull} \\
    冲突状态 & \texttt{git pull} 提示冲突 & 解决冲突后提交 \\
    分支游离 & \texttt{HEAD detached} 警告 & 创建临时分支保存 \\
    \bottomrule
\end{tabular}
\end{center}

\subsection{多场景应用示例}
\subsubsection{场景1:新功能开发}
\begin{verbatim}
# 创建登录功能分支
git checkout -b feature/login
git push -u origin feature/login

# 开发并提交
echo "function login(){}" > login.js
git add login.js
git commit -m "实现登录函数"
git push

# 创建PR/MR合并请求(GitHub/GitLab)
\end{verbatim}

\subsubsection{场景2:紧急热修复}
\begin{verbatim}
# 基于生产分支创建修复分支
git checkout production
git pull
git checkout -b hotfix/auth-bug

# 修复并推送
vim auth.js
git commit -am "修复权限验证漏洞"
git push -u origin hotfix/auth-bug

# 立即部署到生产环境
\end{verbatim}

\subsubsection{场景3:团队协作开发}
\begin{verbatim}
# 获取队友新建的远程分支
git fetch origin
git checkout -b feature/payment origin/feature/payment

# 协作开发
git add payment.js
git commit -m "添加支付处理逻辑"
git push

# 同步队友提交
git pull
\end{verbatim}

\subsection{最佳实践}
\begin{itemize}[leftmargin=*, nosep]
    \item \textbf{命名规范}:使用 \texttt{feature/}、\texttt{fix/} 等前缀
    \item \textbf{及时推送}:每日工作结束前推送代码
    \item \textbf{分支清理}:合并后删除远程分支:\\
    \texttt{git push origin --delete feature/login}
    \item \textbf{权限控制}:保护主分支,强制Code Review
    \item \textbf{钩子应用}:配置 \texttt{pre-push} 钩子运行测试
\end{itemize}

\subsection{故障排除}
\begin{enumerate}[leftmargin=*, nosep]
    \item \textbf{推送失败(无权限)} \\
    \texttt{\$ git remote -v} \quad \textcolor{gray}{// 检查远程地址} \\
    \texttt{\$ git remote set-url origin https://<token>@github.com/user/repo.git}
    
    \item \textbf{追踪关系丢失} \\
    \texttt{\$ git branch -u origin/feature/login}
    
    \item \textbf{误删远程分支} \\
    \texttt{\$ git reflog} \quad \textcolor{gray}{// 查找分支最后提交} \\
    \texttt{\$ git push origin 8a3f5e9:feature/login} \quad \textcolor{gray}{// 恢复分支}
\end{enumerate}


\section{远程分支删除操作指南}
\textbf{一句话总结:}  
使用 \texttt{git push} 或 \texttt{git branch} 命令可安全删除远程分支,配合 \texttt{--delete} 参数精确控制清理范围,保持仓库整洁。

\subsection{删除方法详解}
\begin{enumerate}[leftmargin=*, nosep]
    \item \textbf{标准删除命令} \\
    {\color{red}\texttt{\$ git push origin --delete feature/login} }\\
    \textcolor{gray}{→ 删除远程的 feature/login 分支}
    
    \item \textbf{等效简写命令} \\
    \texttt{\$ git push origin :feature/login} \\
    \textcolor{gray}{→ 冒号语法等效于 --delete}
    
    \item \textbf{批量删除分支} \\
    \texttt{\$ git push origin --delete feat/old hotfix/test} \\
    \textcolor{gray}{→ 同时删除多个分支}
\end{enumerate}

\subsection{操作流程图示}
\begin{center}
\begin{tikzpicture}[node distance=1.5cm]
\node (local) [draw] {本地仓库};
\node (remote) [draw, right of=local] {远程仓库};
\node (branch) [draw, above of=remote] {feature/login};

\draw [->] (local) -- node[above] {删除指令} (remote);
\draw [->, red] (branch) -- (remote);
\draw [red] (2.5,1.5) circle (0.3) node {\texttt{X}};
\end{tikzpicture}
\end{center}

\subsection{删除前验证步骤}
\begin{enumerate}[leftmargin=*, nosep]
    \item \textbf{检查分支状态} \\
    {\color{red}\texttt{\$ git branch -a}}   \quad \textcolor{gray}{// 确认分支存在}
    \begin{verbatim}
  main
* feature/login
  remotes/origin/feature/login
    \end{verbatim}
    
    \item \textbf{确认合并状态} \\
    \texttt{\$ git log origin/main..origin/feature/login} \\
    \textcolor{gray}{→ 检查是否有未合并的提交}
    
    \item \textbf{权限验证} \\
    \textcolor{gray}{// 确保有远程仓库删除权限}
\end{enumerate}

\subsection{删除后清理操作}
\begin{center}
\begin{tabular}{@{}ll@{}}
    \toprule
    \textbf{命令} & \textbf{作用} \\
    \midrule
    \texttt{git fetch -p} & 清除本地缓存的远程分支记录 \\
    \texttt{git branch -D feature/login} & 删除本地关联分支 \\
    \texttt{git remote prune origin} & 等效于 \texttt{git fetch -p} \\
    \bottomrule
\end{tabular}
\end{center}

\subsection{特殊场景处理}
\begin{enumerate}[leftmargin=*, nosep]
    \item \textbf{恢复误删分支} \\
    \texttt{\$ git reflog} \quad \textcolor{gray}{// 查找分支最后提交} \\
    \texttt{\$ git push origin 8a3f5e9:refs/heads/feature/login}
    
    \item \textbf{删除受保护分支} \\
    \textcolor{gray}{→ 需在GitLab/GitHub设置中取消分支保护}
    
    \item \textbf{无权限删除} \\
    \textcolor{gray}{→ 联系仓库管理员执行删除}
\end{enumerate}

\subsection{最佳实践}
\begin{itemize}[leftmargin=*, nosep]
    \item \textbf{命名规范}:使用前缀如 \texttt{feat/}、\texttt{fix/} 便于识别
    \item \textbf{自动化清理}:配置CI/CD自动删除合并后分支
    \item \textbf{定期维护}:每月执行批量清理 \\
    \texttt{git branch -r | grep 'origin/feat/' | sed 's/origin\///' | xargs git push origin --delete}
    
    \item \textbf{权限控制}:限制成员删除主分支权限
    \item \textbf{备份机制}:重要分支删除前创建标签 \\
    \texttt{git tag archive/feature/login-v1 origin/feature/login}
\end{itemize}

\subsection{GUI工具操作}
\begin{itemize}[leftmargin=*, nosep]
    \item \textbf{GitHub}:仓库页面 → Branches → 分支右侧垃圾桶图标
    \item \textbf{GitLab}:仓库 → Branches → 分支右侧删除按钮
    \item \textbf{VS Code}:分支视图 → 远程分支右键"Delete Branch"
    \item \textbf{GitKraken}:远程分支右键"Delete origin/feature/login"
\end{itemize}

\subsection{删除状态验证}
\begin{verbatim}
# 删除前
$ git ls-remote origin
8a3f5e9...refs/heads/feature/login

# 删除后
$ git ls-remote origin
[feature/login 分支消失]
\end{verbatim}