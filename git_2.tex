\section{Git项目提交远程仓库全流程指南}
\textbf{核心要义:}
通过五步标准化操作完成本地项目到GitHub的完整提交,涵盖仓库初始化、文件追踪、分支管理、远程关联与推送策略,解决常见错误场景。

\subsection{基础操作流程}
\begin{enumerate}[leftmargin=*, nosep]
\item \textbf{初始化本地仓库}\\
创建项目目录并进入,执行初始化命令:\\
\texttt{mkdir project-name \&\& cd project-name} \\
\texttt{git init}\quad \textit{\# 生成.git隐藏目录}

\item \textbf{添加文件到暂存区}\\
创建项目文件后执行追踪操作:\\
\texttt{touch README.md} \quad \textit{\# 创建示例文件} \\
\texttt{git add .}\quad \textit{\# 添加所有文件(或指定文件名)}

\item \textbf{提交到本地仓库}\\
记录版本快照并添加描述:\\
\texttt{git commit -m "初始化项目"}\quad \textit{\# 提交信息需明确}

\item \textbf{关联远程仓库}\\
复制GitHub仓库HTTPS/SSH链接后执行:\\
\texttt{git remote add origin https://github.com/HR689hHN/Computer.git}

\item \textbf{推送到远程分支}\\
首次推送需建立分支关联:\\
\texttt{git push -u origin main}\quad \textit{\# -u设置默认上游分支}
\end{enumerate}

\subsection{关键配置说明}
\begin{enumerate}[leftmargin=*, nosep]
\item \textbf{身份信息配置}\\
首次使用Git必须设置全局身份(与GitHub一致):\\
\texttt{git config --global user.name "YourName"} \\
\texttt{git config --global user.email "email@example.com"}

\item \textbf{分支命名规范}\\
现代Git默认使用\texttt{main}分支(历史用\texttt{master}):\\
\texttt{git branch -m master main} \quad \textit{\# 旧分支重命名}

\item \textbf{文件忽略规则}\\
创建\texttt{.gitignore}过滤非追踪文件:
\begin{verbatim}
# 示例内容
node_modules/
.env
*.log
\end{verbatim}
\end{enumerate}

\subsection{高级场景处理}
\begin{enumerate}[leftmargin=*, nosep]
\item \textbf{推送冲突解决}\\
当远程有本地缺失的提交(如初始化README):\\
\texttt{git pull origin main --rebase} \quad \textit{\# 变基合并} \\
\texttt{git push origin main}

\item \textbf{远程关联错误修正}\\
"origin已存在"时的处理方案:\\
\texttt{git remote set-url origin 新URL} \quad \textit{\# 更新仓库地址} \\
\texttt{git remote rm origin} \quad \textit{\# 删除后重新添加}

\item \textbf{空仓库推送策略}\\
本地无提交内容时需先创建初始提交:\\
\texttt{echo "\# Project" > README.md} \\
\texttt{git add . \&\& git commit -m "init"}
\end{enumerate}

\subsection{操作验证命令}
\begin{tabular}{ll}
\textbf{查看仓库状态} & \texttt{git status} \\
\textbf{检查远程关联} & \texttt{git remote -v} \\
\textbf{验证分支关联} & \texttt{git branch -vv} \\
\textbf{查看提交历史} & \texttt{git log --oneline} \\
\end{tabular}

\vspace{1em}
\textbf{最佳实践:} 推送前执行\texttt{git status}确认无未跟踪文件,使用\texttt{git push}前优先\texttt{git pull --rebase}避免冲突,敏感信息务必写入\texttt{.gitignore}。


\section{后续修改提交远程仓库标准流程}
\textbf{核心策略:}
基于主分支(main/master)直接提交或创建特性分支并行开发,通过四步标准化操作实现高效更新,匹配不同协作场景需求。

\subsection{分支决策模型}
\begin{enumerate}[leftmargin=*, nosep]
\item \textbf{直接主分支提交} \\
\textit{适用场景:个人项目/紧急修复/微小变更}\\
流程:工作目录修改 → 暂存 → 提交 → 推送\\
优势:路径最短,适合独立开发者

\item \textbf{创建特性分支提交} \\
\textit{适用场景:团队协作/功能开发/长期修改}\\
流程:创建分支 → 开发 → 本地测试 → 推送分支 → PR合并\\
优势:隔离风险,支持并行开发
\end{enumerate}

\subsection{主分支直接提交流程}
\begin{enumerate}[leftmargin=*, nosep]
\item \textbf{修改工作文件} \\
在项目目录编辑代码文档:\\
\texttt{vim index.html} \quad \textit{\# 或IDE可视化编辑}

\item \textbf{查看变更状态} \\
确认修改范围及内容:\\
\texttt{git status} \quad \textit{\# 显示红标未暂存文件} \\
\texttt{git diff} \quad \textit{\# 查看具体代码变动}

\item \textbf{暂存并提交变更} \\
选择需提交的修改:\\
\texttt{git add .} \quad \textit{\# 添加所有修改} \\
\texttt{git add src/} \quad \textit{\# 添加特定目录} \\
\texttt{git commit -m "修复登录页面样式异常"}

\item \textbf{推送到远程仓库} \\
同步至云端(首次后无需-u参数):\\
\texttt{git push origin main} \quad \textit{\# 关联后简化为 git push}
\end{enumerate}

\subsection{特性分支开发流程}
\begin{enumerate}[leftmargin=*, nosep]
\item \textbf{创建开发分支} \\
基于主分支新建工作分支:\\
\texttt{git checkout -b feat/user-profile} \\
命名规范:\texttt{feat/}功能 \texttt{fix/}修复 \texttt{docs/}文档

\item \textbf{分支内迭代开发} \\
在独立分支完成修改:\\
\texttt{git add . \&\& git commit -m "增加头像上传功能"} \\
\textit{※ 可多次提交形成开发历史}

\item \textbf{推送到远程分支} \\
上传分支至GitHub:\\
\texttt{git push -u origin feat/user-profile}

\item \textbf{合并到主分支} \\
通过PR/MR完成集成:\\
\begin{itemize}[leftmargin=*]
\item GitHub:创建 Pull Request
\item GitLab:创建 Merge Request
\item 命令行合并:\texttt{git checkout main \&\& git merge feat/user-profile}
\end{itemize}
\end{enumerate}

\subsection{关键注意事项}
\begin{tabular}{p{0.2\textwidth}p{0.75\textwidth}}
\textbf{冲突预防} & 推送前执行 \texttt{git pull --rebase} 避免版本冲突 \\
\textbf{原子提交} & 单次提交仅完成单一功能(避免混合修改) \\
\textbf{信息规范} & 提交消息格式:\texttt{<类型>(<范围>): <描述>} \\
\textbf{分支清理} & 合并后删除本地分支:\texttt{git branch -d feat/user-profile} \\
\textbf{敏感防护} & 修改内容若含密钥需重置历史:\texttt{git filter-repo --force} \\
\end{tabular}

\vspace{1em}
\textbf{操作效率工具链:}
\begin{itemize}[leftmargin=*, nosep]
\item \texttt{git stash}:临时保存未完成修改
\item \texttt{git restore}:撤销工作区修改
\item \texttt{.gitignore}:配置忽略文件模板
\item GitHub Desktop:可视化分支管理
\end{itemize}
