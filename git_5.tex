\section{误删分支恢复操作指南}
\textbf{一句话总结:}  
通过Git的引用日志和分支追踪机制,可高效恢复误删分支,核心步骤为:定位历史提交 → 重建分支指针 → 同步远程仓库。

\subsection{恢复流程详解}
\begin{enumerate}[leftmargin=*, nosep]
    \item \textbf{查找分支最后提交} \\
    \texttt{\$ git reflog} \quad \textcolor{gray}{// 显示所有操作历史}
    \begin{verbatim}
8a3f5e9 (HEAD -> main) HEAD@{0}: checkout: moving from feature/login to main
d2b4c6c HEAD@{1}: commit: 用户登录优化
8a3f5e9 HEAD@{2}: checkout: moving from main to feature/login
    \end{verbatim}
    
    \item \textbf{重建本地分支} \\
    \texttt{\$ git branch feature/login d2b4c6c} \quad \textcolor{gray}{// 基于提交哈希重建}
    
    \item \textbf{验证分支内容} \\
    \texttt{\$ git checkout feature/login} \\
    \texttt{\$ git log --oneline} \quad \textcolor{gray}{// 确认提交历史完整}
    
    \item \textbf{恢复远程分支} \\
    \texttt{\$ git push -u origin feature/login}
\end{enumerate}

\subsection{恢复原理图示}
\begin{center}
\begin{tikzpicture}[node distance=1.5cm]
\node (reflog) [draw] {引用日志};
\node (commit) [draw, right of=reflog] {提交记录};
\node (local) [draw, below of=commit] {本地分支};
\node (remote) [draw, below of=local] {远程分支};

\draw [->, blue] (reflog) -- node[right] {1. 定位} (commit);
\draw [->, blue] (commit) -- node[right] {2. 重建} (local);
\draw [->, blue] (local) -- node[right] {3. 推送} (remote);
\end{tikzpicture}
\end{center}

\subsection{关键恢复技术}
\subsubsection{1. 引用日志查询}
\begin{itemize}[leftmargin=*, nosep]
    \item \texttt{git reflog}:显示所有HEAD变更记录
    \item \texttt{git reflog show --all}:显示所有引用变更
    \item 时间过滤:\texttt{git reflog --since="2 days ago"}
\end{itemize}

\subsubsection{2. 分支重建方法}
\begin{center}
\begin{tabular}{@{}ll@{}}
    \toprule
    \textbf{场景} & \textbf{命令} \\
    \midrule
    已知提交哈希 & \texttt{git branch <分支名> <commit\_hash>} \\
    最后操作记录 & \texttt{git branch <分支名> HEAD@\{1\}} \\
    远程分支恢复 & \texttt{git push origin <commit\_hash>:refs/heads/<分支名>} \\
    \bottomrule
\end{tabular}
\end{center}

\subsubsection{3. 恢复验证}
\begin{itemize}[leftmargin=*, nosep]
    \item 提交历史:\texttt{git log -5 --oneline}
    \item 文件验证:\texttt{git show HEAD:filename}
    \item 分支关联:\texttt{git branch -vv}
\end{itemize}

\subsection{特殊场景处理}
\begin{enumerate}[leftmargin=*, nosep]
    \item \textbf{引用日志被清除} \\
    \texttt{\$ git fsck --full} \quad \textcolor{gray}{// 检查悬空对象} \\
    \texttt{\$ git show d2b4c6c} \quad \textcolor{gray}{// 手动验证提交}
    
    \item \textbf{仅删除远程分支} \\
    \texttt{\$ git checkout -b feature/login} \\
    \texttt{\$ git push -u origin feature/login}
    
    \item \textbf{分支名称遗忘} \\
    \texttt{\$ git log --all --grep="登录功能"} \quad \textcolor{gray}{// 按提交信息搜索}
\end{enumerate}

\subsection{预防措施}
\begin{itemize}[leftmargin=*, nosep]
    \item \textbf{分支保护}:
    \begin{itemize}[leftmargin=*, nosep]
        \item GitHub/GitLab设置保护分支
        \item \texttt{git config --global branch.autosetupmerge always}
    \end{itemize}
    
    \item \textbf{定期备份}:
    
\begin{itemize}[leftmargin=*, nosep]
        \item 创建存档标签:\texttt{git tag archive/feature/login-v1 feature/login}
        \item 推送备份仓库:\texttt{git push backup-repo --all}
    \end{itemize}
    
    \item \textbf{操作规范}:
    
\begin{itemize}[leftmargin=*, nosep]
        \item 使用 \texttt{git branch -d} 而非 \texttt{-D}
        \item 删除前执行:\texttt{git merge --no-ff feature/login}
    \end{itemize}
    
    \item \textbf{工具支持}:
    
\begin{itemize}[leftmargin=*, nosep]
        \item IDE分支管理(VSCode/GitKraken)
        \hline
        \item 日志可视化:\texttt{git log --graph --all --oneline}
    \end{itemize}
\end{itemize}

\subsection{恢复工作流示例}
\begin{verbatim}
# 场景:误删feature/login分支
$ git branch -D feature/login

# 步骤1:查询操作记录
$ git reflog
d2b4c6c HEAD@{3}: commit: 用户登录优化
8a3f5e9 HEAD@{4}: checkout: moving from main to feature/login

# 步骤2:重建分支
$ git branch feature/login d2b4c6c

# 步骤3:验证内容
$ git checkout feature/login
$ ls login.js  # 确认关键文件存在

# 步骤4:恢复远程
$ git push -u origin feature/login

# 步骤5:确认恢复
$ git branch -a
  feature/login
  remotes/origin/feature/login
\end{verbatim}

\subsection{图形化工具操作}
\begin{itemize}[leftmargin=*, nosep]
    \item \textbf{VS Code}:
    \begin{itemize}[leftmargin=*, nosep]
        \item GitLens插件 → REFERENCE HISTORY
        \item 右键提交记录 → "Create Branch..."
    \end{itemize}
    
    \item \textbf{GitKraken}:
    
\begin{itemize}[leftmargin=*, nosep]
        \item 左侧菜单 → REFLOG
        \item 右键提交 → "Create branch here"
    \end{itemize}
    
    \item \textbf{Sourcetree}:
    
\begin{itemize}[leftmargin=*, nosep]
        \item 视图 → "Reflog"
        \item 右键记录 → "创建分支"
    \end{itemize}
\end{itemize}